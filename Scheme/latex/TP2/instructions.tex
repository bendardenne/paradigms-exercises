\documentclass{../../../tp}


\title{Practical Session 4: \scheme (Functional programming)}
\author{}

\begin{document}
	
	\maketitle



\section{Functional Programming}

\scheme follows the functional programming paradigm. To put it simply, functional programming is built around the concept of \emph{functions}, viewed as mathematical objects: a function takes some inputs and returns an output, with no other side-effects. Functional programming avoids mutation of data: an argument passed to a function is never modified itself, it's a new object which is returned as the result of applying a function to an argument. In functional programming, a program is viewed as a series of functions applied to input data.

\section{First-class functions}

Like many other functional programming languages, \scheme has first-class functions. This means that functions can be treated like any other entities and the program, and more specifically, a function can be passed as argument to, can be returned from a function and can stored in a variable.

\begin{instruction}
	Write a \schemecode{(filter p l)} function. This function takes two arguments: the first is a predicate $p$ and the second is a list $l$. It returns the list $l$, in which all the elements for which $p$ is false have been removed. Consider this example:

\begin{minted}{scheme}
	(define mylist '(1 2 -3 4 5 -6 7 8 -9))
	(filter positive? mylist)
	 
	 -> '(1 2 4 5 7 8)
\end{minted}

\schemecode{positive?} is of course a built-in \scheme predicate which returns \schemecode{#t} if its argument is a positive number. 
\end{instruction}

\begin{instruction}
	In the previous session, you wrote a \schemecode{sum-cubes} function which may have looked like the following: 
	
	\begin{minted}{scheme}
	(define (sum-cubes a b)
	   (if (> a b)
	      0
	      (+ (cube a) (sum-cubes (+ a 1) b))))
	\end{minted}
	
	Rewrite a more generic \schemecode{sum-func} which takes three arguments $f$, $a$ and $b$ and which returns the sum of $f$ applied to all integers in $\{a, a+1, ... b\}$. Uses this function to perform a \schemecode{sum-cubes}, for instance:
	\begin{minted}{scheme}
	(sum-func cube 0 3)
	-> 36
	\end{minted}
	
	Finally, define \schemecode{sum-cubes} in terms of this new \schemecode{sum-func}.
\end{instruction}


\subsection{Anonymous functions}

Sometimes it can be useful to write a small function as argument to another directly in the function call, without \schemecode{define}'ing it entirely. This is where \schemecode{lambda}'s become handy. For instance, to use \schemecode{filter} with a predicate that keeps all items smaller than 5:

\begin{minted}{scheme}
(filter (lambda (x) (< x 5)) mylist)
-> (1 2 -3 4 -6 -9)
\end{minted}

The  \schemecode{lambda} expression creates an anonymous function which, to $x$ associates the boolean $x != 5$. 

\subsection{Map and apply}

\begin{instruction}
	Some functions defined in the \scheme standard take a functional parameter: for instance \schemecode{apply} or \schemecode{map}. They are important functional programming tools which are also found in many modern languages (sometimes under a different name)
	
	\begin{enumerate}
		\item 	Try to guess / understand what each does based on the following examples:
	
	\begin{minted}{scheme}
	(define mylist '(1 2 -3 4 5 -6 7 8 -9))
	
	(map (lambda (x) (+ x 1)) mylist)
	(apply + mylist)
	(apply max mylist)
	\end{minted}

	\item Write a \schemecode{(range a b)} utility function which produces a list of all integers from $a$ to $b$, included. Then use this function to re-implement  \schemecode{sum-cubes} in terms of \schemecode{map} and  \schemecode{apply}.

	\item Write a function which takes a list of lists and returns a list containing the maximum of each list.
	\begin{minted}{scheme}
	
	(define listoflists '((1 2 3 1) (45 1 3 4 5) (4 5 64) 
		(4 6) (144) (0 4 4) (14 464 4 7 6)))
	
	(max-list listoflists)
	-> '(3 45 64 6 144 4 464)
	\end{minted}
	\end{enumerate}
\end{instruction}

Another common operation we might want to do on list is to \emph{accumulate} the result of a function over all the arguments of this list. In many languages, this function is called \schemecode{fold}. However, the \scheme standard does not define this function. 

\begin{instruction}
	Write an \schemecode{accumulate} function. This function will have the following interface :
	\begin{minted}{scheme}
	(accumulate init-value function list)
	\end{minted}	
	
	And its value is the application of the two-parameter \schemecode{function} successively on all elements of \schemecode{list}, with the first parameter being the accumulated value so far, starting with \schemecode{init-value}. In other words, \schemecode{(accumulate x0 f '(l1 l2 ... ln))} should return
	$$ f(l_n, f(l_{n-1}, ... \quad f(l_2, f(l_1, x_0))) $$
	
	For instance :
	\begin{minted}{scheme}
	(accumulate 1 * '(2 3 4))
	> 24
	
	(accumulate 14 + '(2 3 4))
	> 23
	\end{minted}	
	
	Take care of the parameter order ($f(a,b)$ may not be equal to $f(b,a)$! Respect the order given above).
	
	Can you use this function to reverse a list? 
	
	Why can you not easily use this function to implement your \schemecode{collatz} function of the previous weeks? 
\end{instruction}

	
\subsection{Higher-order functions}

As said above, it is possible for a function to return a function (think for example about what the \schemecode{lambda} builtin returns). 

\begin{instruction}
	Write a function \schemecode{mean} which takes as argument a numeric function $f: \mathbb{R} \rightarrow \mathbb{R}$ and returns the function: 
	
	$$ f': \mathbb{R} \times \mathbb{R} \rightarrow \mathbb{R} \qquad  x,y  \mapsto  \frac{f(x) + f(y)}{2}$$
	
	
	Use this function with the \schemecode{cube} function to generate a \schemecode{mean-cube} function. Then compute the mean-cube of 4 and 16.
	
	\begin{minted}{scheme}
	(mean-cube 4 16)
	-> 2080
	\end{minted}
	
\end{instruction}

\section{Special forms and parameter evaluation}

From the syntax, it looks like \schemecode{if} in \scheme is just another function of three arguments: \schemecode{(if condition true-expr false-expr)}. However, we will see that there is a slight caveat to this approach and that, under the hood, things are a bit different.

\begin{instruction}
	Implement a \schemecode{(my-if condition true-expr false-expr)} function in terms of \schemecode{cond}. Try it with the following expressions:
	
	\begin{minted}{scheme}
	(define x 4)
	(my-if (positive? x) x (- x))
	
	(my-if (positive? x) 
		(display "positive") 
		(display "negative"))
	\end{minted}
\end{instruction}

What goes on in your \schemecode{my-if} function is that the interpreter first evaluates \emph{all} the parameters passed to the the function. In the first example, this does not matter, because \schemecode{x} and \schemecode{(- x)} simply evaluate to 4 and -4, without any side-effects. But in the second example, both arguments have the side-effect of displaying some text on the output, and their evaluation produces \schemecode{#<void>} (this is an implementation choice in racket. The standard simply specifies that \schemecode{display}, returns ``an unspecified value''). In other words, both the true and false blocks are evaluated upon parameter passing, and the function call actually becomes \schemecode{(my-if #t #<void> #<void>)} and the \schemecode{display} have already printed out their arguments.  

\subsection{Hygienic macros}

Because of this, some constructs in \scheme are actually \emph{special forms} and not functions. In a special form, the arguments are only evaluated when called within the special form, and not when passed as arguments. It is possible to define your own special forms using \schemecode{define-syntax} and \schemecode{syntax-rules}. You can refer to the R5RS\footnote{\url{http://www.schemers.org/Documents/Standards/R5RS/}} for a reference, or some tutorials available online\footnote{See for instance \url{http://www.willdonnelly.net/blog/scheme-syntax-rules/}} for help on how to use these constrcuts.

\begin{instruction}
Rewrite \schemecode{my-if} using a hygienic macro. Do you observe the problem discovered above? 	
\end{instruction}

\end{document}