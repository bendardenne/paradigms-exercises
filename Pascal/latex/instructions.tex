\documentclass{../../tp}


\title{Practical Session 2: \pascal}
\author{}

\begin{document}

\maketitle


\section{Getting started}

Download and install \textsf{FPC}, the \textsf{Free Pascal Compiler}\footnote{ \url{http://www.freepascal.org/download.var}}.\textsf{FPC} is a \pascal standalone command-line compiler. To compile a \pascal source, use: 

\verb|$ fpc sourceCode.pas|

which should generate a \verb|sourceCode.o| object file, as well as a \verb|sourceCode| executable binary.

\subsection{Error: Can't find unit}

If the compiler complains that it can't find a unit with an error message \verb|Fatal: Can't find unit unitName|, it probably means that your compiler is not configured to find the unit packages properly.

The first solution for this is to avoid using units entirely. You should not need them for these practical sessions.

If you really want to use units: locate the \verb|fpc.cfg| config file (on Linux, try \verb|.fpc.cfg| in your home directory) and check that the path to your units installation is included. Typically, on a Linux system, you should have a line with:

\verb| -Fu/usr/lib/fpc/$fpcversion/units/$fpctarget/*|

\subsection{Linker warning}

The linker may give the error:

\verb|warning: link.res contains output sections; did you forget -T?|

It can be safely ignored. 

\subsection{Using the compiler}
\begin{instruction}
	Try to compile and run the following \emph{Hello, World!} program.
	
	\begin{lstlisting}[language=Pascal]
	program HelloWorld;
	begin 
		writeln('Hello, World!');
	end.
	\end{lstlisting}
\end{instruction}

\section{Imperative programs}

\subsection{Flow Control}

\pascal is an imperative programming language, like \C for instance. As such, it offers the usual common idioms for loops, conditions etc. 

\begin{instruction}

Rewrite the AbsMean program of the Assembly session in \pascal. 

\end{instruction}

\subsection{Procedures and functions}
\begin{instruction}
	Write a \pascal function which implements the following function $f$:
		
	\begin{equation*}
		f(n) = 
			\begin{cases}
				n/2 & \text{if $n$ is even} \\
				3n + 1 & \text{if $n$ is odd} 
			\end{cases}
	\end{equation*}
\end{instruction}

\subsection{Recursion}
\begin{instruction}
Write a program that queries the user for a number and checks the Collatz conjecture for this number.

The Collatz conjecture states that for any starting number,  if we repeatedly apply the $f$ function defined above, we will always eventually reach $1$. Use recursion to apply $f$ until reaching 1.

Your program should print every step of the convergence. Example of output, starting with $n = 13$: 
\begin{verbatim}
$ ./collatz
Input a positive integer:
13
13 40 20 10 5 16 8 4 2 1
\end{verbatim}

Write two solutions : one using recursion and one using an iterative process.

\end{instruction}


\section{Data Types}

\pascal provides nice features for user-defined data types. 

Enumerations allow you to enforce a certain defined value for a variable. Only enumerated values can be assigned to variable typed as an enumeration.

\begin{instruction}
	
Define an enumeration type \pascalcode{booktype}. A book can be either a \pascalcode{novel} or a \pascalcode{comic}.

\end{instruction}

\pascal allows programmers to define heterogeneous data structures called \pascalcode{record}s (think of \C's structs) containing multiple fields of different types, including other records or enumerations.

\begin{instruction}
Write a program which defines a \pascalcode{record} for \pascalcode{book}s in a library. Each \pascalcode{book} should have a \emph{title} and an \emph{author} (\pascalcode{string}), a publication date (define an auxiliary \pascalcode{type date}) as well as a \emph{kind} (\pascalcode{booktype}).
\end{instruction}

An important feature of \pascal is variant records. A variant record is a record in which some of the fields are dependent on the value of one specific field. In other words, the variant fields have a different type depending on their use. This feature of \pascal can be abused to circumvent \pascal's static, strong typing, and allow the programmer to access the same memory location as different types. 

\begin{instruction}
Modify your \pascalcode{book record} to have additional records for novels and comic books. More precisely:
	\begin{itemize}
		\item if the book is a novel, it should have an \emph{author} (\pascalcode{string}) and a ISBN number (\pascalcode{Int64}).
		\item if it's a comic book, it should have a \emph{writer} (for the scenario and dialogue) and an \emph{artist} (for the drawings). Both are of course \pascalcode{string}s.
	\end{itemize} 
	
	Try to play around with those data structures. What happens if you try to assign a value to the writer field of a novel, or the author field of a comic book? What happens if you read that value?
	Specifically, try to run the following and understand what is happening:
	
\begin{lstlisting}[language=Pascal]
theHobbit.title := 'The Hobbit, or There and Back Again';
theHobbit.author := 'J. R. R. Tolkien';
theHobbit.ISBN := 9780007525492;

luckyLuke.kind := comic;
luckyLuke.title := 'Billy the Kid';
luckyLuke.writer := 'Goscinny';
luckyLuke.artist := 'Morris';


{* Tests *}
writeln(theHobbit.kind);
writeln(theHobbit.author);
writeln(theHobbit.writer);
writeln(theHobbit.artist);

writeln(luckyLuke.kind);        
writeln(luckyLuke.author);

luckyLuke.author := 'Did I break Pascal yet?';
writeln(luckyLuke.author);
writeln(luckyLuke.writer);

\end{lstlisting}

	What does the garbage printed by \pascalcode{theHobbit.artist} correspond to?
	What happens if you set the ISBN to 31084784824439585 and print the artist field again? 
\end{instruction}

\end{document}
	